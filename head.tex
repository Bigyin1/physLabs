\documentclass[a4paper, 12pt]{article} % тип документа

%%%Библиотеки
%\usepackage[warn]{mathtext}	
\usepackage[T2A]{fontenc}   %Кодировка
\usepackage[utf8]{inputenc} %Кодировка исходного текста
\usepackage[english, russian]{babel} %Локализация и переносы
\usepackage{caption}
\usepackage{listings}
\usepackage{amsmath, amsfonts, amssymb, amsthm, mathtools}
\usepackage[warn]{mathtext}
\usepackage[mathscr]{eucal}
\usepackage{wasysym}
\usepackage{graphicx} %Вставка картинок правильная
\DeclareGraphicsExtensions{.pdf,.png,.jpg}
\graphicspath{ {images/} }

\setlength{\parskip}{0.5cm}

\usepackage{pgfplots}
\usepackage{indentfirst}
\usepackage{float}    %Плавающие картинки
\usepackage{wrapfig}  %Обтекание фигур (таблиц, картинок и прочего)
\usepackage{fancyhdr} %Загрузим пакет
\usepackage{lscape}
\usepackage{xcolor}
\usepackage[normalem]{ulem}
\usepackage{wasysym}
\usepackage{subfig}
\usepackage{graphicx}
\usepackage[ampersand]{easylist}


\usepackage{titlesec}
\titlelabel{\thetitle.\quad}

\usepackage{hyperref}
\newenvironment{comment}{}{}

%%%Конец библиотек

%%%Настройка ссылок
%%%	\hypersetup
%%%	{
%%%		colorlinks = true,
%%%		linkcolor  = blue,
%%%		filecolor  = magenta,
%%%		urlcolor   = blue
%%%	}
%%%Конец настройки ссылок


%%%Настройка колонтитулы
\pagestyle{fancy}
\fancyhead{}
\fancyhead[L]{1.1.4}
\fancyhead[R]{Апарин Сергей, группа Б01-205}
\fancyfoot[C]{\thepage}
%%%конец настройки колонтитулы


\usepackage[T2A]{fontenc}                        % кодировка
\usepackage[utf8]{inputenc}                        % кодировка исходного текста
\usepackage[english,russian]{babel}        % локализация и переносы
\usepackage{tikz}
\usepackage{pgfplots}
