\section{Ход работы}
\subsection{Изучение работы оптического пирометра}

\textit{С помощью пирометра измеряется температура модели АЧТ и проводится сравнение её значения  со значением температуры, измеренной при помощи термопарного термометра.}
\begin{enumerate}
    \item Настроим пирометр, прогреем его нить. Прогреем модель АЧТ.
    \item Введём красный светофильтр пирометра. Изменяя ток через нить пирометра, добьёмся исчезновения нити на фоне изображения раскалённой поверхности дна АЧТ. Проверим корректность измерений: температура на пирометре не должна сильно отличаться от температуры АЧТ, измеренной термопарой. Результаты измерений занесём в таблицу 1.

          \begin{table}[h]
              \centering
              \begin{center}
                  \caption{Сравнение температуры нити пирометра и температуры АЧТ}
              \end{center}
              \vspace{0.1cm}
              \label{tab:my_label}
              \begin{tabular}{ |p{4.5cm}||p{1cm}|p{1cm}|p{1cm}|p{1cm}|}
                  \hline
                  Температура на пирометре, $^{\circ}$C & 880  & 876  & 860  & 855  \\
                  \hline
                  Напряжение на термопаре, мВ           & 40.1 & 39.6 & 39.0 & 38.2 \\
                  \hline
                  Температура АЧТ, $^{\circ}$C          & 970  & 960  & 940  & 920  \\
                  \hline
                  Разность значений, \%                 & 10   & 9.3  & 8.8  & 7.2  \\
                  \hline
              \end{tabular}
          \end{table}

          Видим, что разница в показаниях приборов не превышает 10\%
\end{enumerate}


\subsection{Измерение яркостной температуры тел}
\textit{Показывается, что разные тела, накалённые до одинаковой термодинамической температуры, имеют различную яркостную температуру}
\begin{enumerate}
    \item Прогреем керамическую трубку с образцами до красного каления.
    \item Измерим яркостную температуру поверхности трубки и колец:
          \begin{center}
              Керамическая трубка \hspace{1cm} 750-800 $^{\circ}$C на разных участках \\
              Металлические кольца \hspace{1cm} $\approx$ 700 $^{\circ}$C \\
              Неметаллические кольца \hspace{1cm} $<$ 700 $^{\circ}$C (не определяется пирометром)
          \end{center}
          Несовпадение яркостной температуры у различных тел, имеющих одинаковую термодинамическую температуру, вытекает из того, что эти две величины связаны, в том числе, через спектральный коэффициент поглощения, который у разных материалов различный.
\end{enumerate}

\item Представим зависимость $W=f(T)$ в логарифмическом масштабе как $\ln(W) = \ln(\varepsilon_T \sigma S) + n \ln(T)$ По углу наклона графика можно определить показатель степени температуры в законе Стефана-Больцмана. Он получился примерно равен 3, при теоретическом значении 4.

\begin{figure}[h]
    \centering
    \includegraphics[width=\textwidth]{Graph2.png}
    \caption{Зависимость мощности лампы от её термодинамической температуры, логарифмический масштаб}
    \label{fig:vac}
\end{figure}

\item Определим постоянную Стефана-Больцмана, используя значение термодинамической температуры 1870$^{\circ}$C и соответствующую мощность ($\varepsilon_T(1870) \approx 0.232$, $S = 0.36$ см$^2$):
\begin{center}
    $\sigma = \frac{W}{\varepsilon_T S T^4} = 0.158 \cdot 10^{-12}$ Вт/(см$^2 \cdot$ K$^4$)
\end{center}
Также можно определить постоянную Стефана-Больцмана, используя построенный график зависимости $\ln(W) = \ln(\varepsilon_T \sigma S) + n \ln(T)$
\begin{center}
    $\ln(\varepsilon_t \sigma S) = -22.0016$ \\
    \\
    $\sigma = \frac{e^{-22.0016}}{\varepsilon_T S} = 3.335 \cdot 10^{-12}$ Вт/(см$^2 \cdot$ K$^4$)
\end{center}

Значение постоянной Стефана-Больцмана, определённое первым методом, на порядок больше теоретического; значение же её, определённое по графику, практически совпадает с теоретическим
\begin{center}
    $\sigma_t_h = 5.67\cdot 10^{-12}$ Вт/(см$^2 \cdot$ K$^4$)
\end{center}

\item Оценим значение постоянной Планка:
\begin{center}
    $h = \sqrt[3]{\frac{2 \pi^5 k_B^4}{15 c^2 \sigma}} \approx 10^{-27}$ эрг$\cdot$с
\end{center}

\end{enumerate}

\subsection{Измерение яркостной температуры неоновой лампочки}
Термодинамическая температура неоновой лампочки примерно равна комнатной, и не соответствует её яркостной температуре ($\approx$ 830$^{\circ}$C). Дело в том, что неоновая лампочка в принципе не является моделью абсолютно чёрного или серого тела, и её излучение носит совершенно другую природу (переход электронов между энергетическими уровнями). То, что её свет имеет такой же цвет, что и нагретое АЧТ - совпадение.

\section{Вывод}
В ходе работы было изучено тепловое излучение модели абсолютно чёрного тела и моделей серых тел - колец из различных материалов и вольфрамовой нити. Было проведено ознакомление с принципом работы оптического пирометра - в ходе его настройки и работы с моделью АЧТ выяснилось, что разность показаний пирометра и действительной температурой составляет до 10\%. Этот фактор мог быть причиной того, что в ходе работы не было подтверждено выполнение закона Стефана-Больцмана. \par
При проведении работы мы наблюдали, что для различных материалов с одинаковой термодинамической температурой их яркостная температура может не совпадать. Это связано с различием коэффициента спектрального поглощения этих материалов. \par
В работе было предложено проверить справедливость закона Стефана-Больцмана ($W \propto T^4$). К сожалению, такую зависимость получить не удалось - значение степени при Т, определённое в работе, составляло 2.68 $\approx$ 3. Возможные причины этого несовпадения:
\begin{itemize}
    \item сильный теплоотвод от нити
    \item ошибка в показаниях пирометра
    \item ошибка в визуальном определении яркости
\end{itemize}
Также по результатам измерений была оценена постоянная Стефана-Больцмана двумя способами - непосредственно используя формулу (2) и используя график зависимости $W(t)$ в логарифмическом масштабе. ��торой способ оказался более точным:
\begin{center}
    $\sigma_1 = 0.158\cdot 10^{-12}$ Вт/(см$^2 \cdot$ K$^4$) \\
    $\sigma_2 = 3.335 \cdot 10^{-12}$ Вт/(см$^2 \cdot$ K$^4$) \\
    $\sigma_t_h = 5.67\cdot 10^{-12}$ Вт/(см$^2 \cdot$ K$^4$)
\end{center}
Наконец, в ходе работы с помощью пирометра была определена "яркостная температура" неоновой лампочки, не являющейся моделью АЧТ. Эта яркостная температура не совпадает с термодинамической







\end{document}
