\section*{Цель работы}
Исследовать явления дифракции Френеля и Фраунгофера на щели, изучить влияние дифракции на разрешающую способность оптических инструментов.


\section*{В работе используются}
Оптическая скамья, ртутная лампа, монохроматор, щели с регулируемой шириной, рамка с вертикальной нитью, двойная щель, микроскоп на поперечных салазках с микрометрическим винтом, зрительная труба.


\section*{Теоретическая справка}
\paragraph{Дифракция Френеля}
Суммарное ширина $n$ зон Френеля $z_n$ определяется соотношением
\begin{equation}
    z_n = \sqrt{an\lambda}
\end{equation}
где $n$ -- номер зоны, $a$ -- расстояние от щели до плоскости наьлюдений, $\lambda$ -- длина волны. В работе используется ртутная лампа с длиной волны $\lambda = 5461 \cdot 10^{-10}$ м.

\paragraph{Дифракция Фраунгофера}
При дифракции Фраунгофера на одной щели имеет место соотношение
\begin{equation}
    X_m = f_2 m \frac{\lambda}{D}
\end{equation}
где $X_m$ -- расстояние темной полосы от оптическое оси объектива, $f_2$ -- фокусное расстояние линзы, $m$ -- номер темной полосы, $D$ -- ширина щели. \\
При дифракции Фраунгофера на двух щелях будем пользоваться соотношением
\begin{equation}
    \delta x = f_2 \frac{\lambda}{d} = \frac{2d}{Dn}
\end{equation}
где $\delta x$ -- линейное расстояние между соседними интерференционными полосами в плоскости наблюдения, $n$ -- число темных полос в центральном максимуме.
